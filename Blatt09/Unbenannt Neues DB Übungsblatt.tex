\documentclass[a4paper,11pt,twoside]{scrartcl}
\usepackage[T1]{fontenc}
\usepackage{subcaption}
\usepackage[utf8]{inputenc}
\usepackage[normalem]{ulem}
\usepackage{ngerman, eucal, mathrsfs, amsfonts, bbm, amsmath, amssymb, stmaryrd,graphicx, array, geometry, color, wrapfig, wasysym}
\geometry{left=25mm, right=15mm, bottom=25mm}
\setlength{\parindent}{0em} 
\setlength{\headheight}{0em} 
\title{Datenbanken\\ Blatt 9}
\subtitle{Gruppe 26}
\author{Markus Vieth\and Christian Stricker}
\date{\today}
\input{../head/lstlisting.tex}
\begin{document}

\newcommand{\cor}[1]{\textcolor{red}{\textit{#1}}}
\maketitle
\cleardoublepage
\pagestyle{myheadings}
\markboth{Markus Vieth, Christian Stricker}{Markus Vieth, Christian Stricker}

\section{Aufgabe}
Zur besseren Lesbarkeit, wird das berechnen der Attributhülle übersprungen.\\
\subsection{}
\begin{align*}
&R = \{ TN, N, A, DTN, DN, MID, MN, Y, L \}\\
&FDs:\\
&TN \rightarrow N, A\\
&DTN \rightarrow DN\\
&MID \rightarrow MN, Y, L\\
&MID, MN, Y, L, TN, N, A \rightarrow DTN, DN 
\end{align*}
\subsection{}
Aus den FDs ergeht folgender Superschlüssel $TN, DTN, MID$, welcher mit der letzten FD zu folgendem Candidate Key (im folgenden CK) gekürzt werden kann: $TN, MID$
\subsection{}
\subsubsection{Linksreduktion}
\begin{align*}
&FDs:\\
&TN \rightarrow N, A\\
&DTN \rightarrow DN\\
&MID \rightarrow MN, Y, L\\
&MID, \overset{\text{überflüssig}}{\text{\sout{MN, Y, L,}}} TN, \overset{\text{überflüssig}}{\text{\sout{N, A}}} \rightarrow DTN, DN 
\end{align*}
\subsubsection{Rechtsreduktion}
\begin{align*}
&FDs:\\
&TN \rightarrow N, A\\
&DTN \rightarrow DN\\
&MID \rightarrow MN, Y, L\\
&MID, TN\rightarrow DTN\overset{\text{überflüssig}}{\text{\sout{DN}}}
\end{align*}
\subsubsection{Canonical Cover}
\begin{align*}
&FDs:\\
&TN \rightarrow N, A\\
&DTN \rightarrow DN\\
&MID \rightarrow MN, Y, L\\
&MID, TN\rightarrow DTN
\end{align*}
\subsection{}
\begin{align*}
R_{Cinema} = &\{\underline{TN}, N, A\}\\
& \text{mit FD :}TN \rightarrow N, A\\
R_{Distributor} = &\{\underline{DTN}, DN\}\\
& \text{mit FD :}DTN \rightarrow DN\\
R_{Movie} = &\{\underline{MID}, MD, Y, L\}\\
& \text{mit FD :}MID \rightarrow MN, Y, L\\
R_{rentedBy} = &\{\underline{TN, MID}, DTN\}\\
& \text{mit FD :}MID, TN\rightarrow DTN
\end{align*}
\section{Aufgabe}
Wir gehen nicht von streng katholischen Familien aus.
\subsection{}
Durch Transitivität, Verstärkung und Verallgemeinerung wären weitere MVDs möglich
\begin{align*}
&R = \{ GF, GM, F, M, C \}\\
&FDs:\\
&C \rightarrow F, M\\
&F, M  \twoheadrightarrow C\\
&F,M  \twoheadrightarrow GF, GM\\
&C \twoheadrightarrow GF, GM\\
&GF, GM \twoheadrightarrow F, M, C\\
&C,GM \rightarrow GF\\
&C,GF \rightarrow GM\\
&F \twoheadrightarrow M\\
&M \twoheadrightarrow F\\
&GF \twoheadrightarrow GM\\
&GM \twoheadrightarrow GF
\end{align*}
\subsection{}
\paragraph{Candidate Keys:} $\{C,GM\}, \{C,GF\}$
\subsection{}
\begin{align*}
& Z = \{\{ GF,GM,F,M,C \}\}\\
& R = \{ GF,GM,F,M,C \}\\
&C \twoheadrightarrow F,M\text{ nicht trivial}\\
&\Rightarrow R_1 = \{ C,F,M \}, R_2 = \{ C, GF, GM \}\\
& Z = \{ R_1, R_2 \} \\
& R_1 = \{ C,F,M \}\\
&F \twoheadrightarrow M\text{ nicht trivial}\\
&\Rightarrow R_{1_1} = \{ F, M\}, R_{1_2} = \{ F, C \}\\
& Z = \{ R_2, R_{1_1}, R_{1_2}\}\\
& R_2 = \{ C, GF, GM \}\\
& GF \twoheadrightarrow GM\text{ nicht trvial}\\
& \Rightarrow R_{2_1} = \{ GF, GM\}, R_{2_2} = \{ GF, C \} \\
& Z = \{ R_{1_1}, R_{1_2}, R_{2_1}, R_{2_2}\}
\end{align*}
Algorithmus ist fertig, da alle weiteren MVDs in den $R_i$ in $Z$ trivial sein müssen, da alle $R_i$ in $Z$ nur noch 2 Elemente haben.\\
Bei Familien ohne Scheidung hätte der Algorithmus mit $R_1$ und $R_2$ geendet.
\end{document}