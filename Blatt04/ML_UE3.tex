\documentclass[a4paper,11pt,twoside]{scrartcl}
\usepackage[T1]{fontenc}
\usepackage{subcaption}
\usepackage[utf8]{inputenc}
\usepackage{ngerman, eucal, mathrsfs, amsfonts, bbm, amsmath, amssymb, stmaryrd,graphicx, array, geometry, color, wrapfig, wasysym}
\geometry{left=25mm, right=15mm, bottom=25mm}
\setlength{\parindent}{0em} 
\setlength{\headheight}{0em} 
\title{Datenbanken\\ Blatt 4}
\subtitle{Gruppe 26}
\author{Markus Vieth\and Christian Stricker}
\date{\today}
\input{../head/lstlisting.tex}
\begin{document}

\newcommand{\cor}[1]{\textcolor{red}{\textit{#1}}}
\maketitle
\cleardoublepage
\pagestyle{myheadings}
\markboth{Markus Vieth, Christian Stricker}{Markus Vieth, Christian Stricker}

\section{Aufgabe}

\subsection{}
\subsubsection{}
\begin{lstlisting}
{[a.Firstname, a.Lastname, d.Firstname, d.Lastname] | a, d $\in$ person
	$\land~\exists$ x $\in$ directs (d.PersonId = x.PersonId
		$\land~\exists$ y $\in$ acts (a.PersonId = y.PersonId
			$\land$ x.MovieId = y.MovieId
		)$\lor~\exists$ x2 $\in$ acts (d.PersonId = x2.PersonId
			$\land~\exists$ y $\in$ acts (a.PersonId = y.PersonId
				$\land$ x2.MovieId = y.MovieId
		)
	) $\land~\neg$ (
		$\neg$(d.Firstname = 'Lars' $\land$ d.Lastname = 'von Trier')
		$\land~\neg$(d.Firstname = 'Michael' $\land$ d.Lastname = 'Haneke')
	)
}

\end{lstlisting}

\subsubsection{}

\begin{lstlisting}
{[c.name, c.location, s2.time, s2.price, m.title] | c $\in$ cinema
	$\land~\exists$ r $\in$ screeningRoom (r.inCinema = c.CinemaId
		$\land~\exists$ s1, s2 $\in$ screening (s1.RoomId = r.RoomId
			$\land~\exists$ m $\in$ movie (s1.MovieId = m.MovieId $\land$ s2.MovieId = m.MovieId
				$\land$ s1.date = Today $\land$ s1.time = '9:00'
			)
		)
	)
}
\end{lstlisting}



\subsubsection{}

\begin{lstlisting}
{[a.Firstname, a.Lastname] | a, d $\in$ person
	$\land~\forall$ m $\in$ movies (d.Firstname = 'Lars' $\land$ d.Lastname = 'von Trier'
		$\land~\exists$ x $\in$ directs (m.MovieId = x.MovieId
			$\land$ d.PersonId = x.PersonId
		) $\Rightarrow$ ( 
			$\exists$ y $\in$ acts (m.MovieId = y.MovieId
				$\land$ a.PersonId = y.PersonId
			)
		)
	) $\land~\exists$ t $\in$ acts (t.PersonId = a.PersonId)
}
\end{lstlisting}
\pagebreak
\subsection{}
\subsubsection{}
\begin{lstlisting}
{ [afn, aln, dfn, dln] | 
	$\exists$ dPid ( [dPid, dfn, dln] $\in$ Person
		$\land$ $\exists$ aPid ([aPid, afn, aln] $\in$ Person
			 $\land$ $\exists$ aMid, r ([aMid, aPid, r] $\in$ acts
				 $\land$ $\exists$ dMid ([dMid, dPid] $\in$ directs
					 $\lor$ $\exists$ role ([dMid, dPid, role] $\in$ acts
						 $\land$ $\neg$(
							$\neg$(dfn = '' $\land$ dln='' ) 
							$\land$ $\neg$(dfn = '' $\land$ dln='' )
						)
					)
				)
			)
		)
	)
}
\end{lstlisting}
\subsubsection{}
\begin{lstlisting}
{ [n, l, time, p, t] | 
	$\exists$ cId ( [cId, n ,l] $\in$ Cinema
		$\land$ $\exists$ rId, num ([rId, num, cId] $\in$ screeningRoom
			$\land$ $\exists$ mId, d, len ([mId, rId, time, d, len, p] $\in$ screening
				$\land$ $\exists$ time2, d2, len2, p2, rId2 ([mId, rId2, time2, d2, len2, p2] $\in$ screening
					$\land$ $\exists$ lang, fsk, y ([mId, t, lang, fsk, y] $\in$ Movie	
						$\land$ time2 = (21*60)
						$\land$ d2 = today
					)
				)
			)
		)
	)
}
\end{lstlisting}
\subsubsection{}
\begin{lstlisting}
{ [afn, aln] |
	 $\exists$ aPid ( [aPid, afn, aln] $\in$ Person
		$\land \forall$  mId ( 
			$\exists$ dPid, dfn, dln ( [dPid, dfn, dln] $\in$ Person
				$\land$ [dPid, mId] $\in$ directs
				$\land$ dfn = 'Vorname'
				$\land$ dln = 'Nachname'
			) $\Rightarrow$ (
				$\exists$ r (
					[aPid, mId, r] $\in$ acts
				)
			)
		) $\land$ $\exists$ mId2, r2  (
			[aPid, mId2, r2] $\in$ acts
		)
	)
}
\end{lstlisting}
\pagebreak
\section{Aufgabe}
\subsection{Beispiel}
\[\{ n | \neg (n \in \text{Professoren}) \}\]
\subsection{Erklärung}
Man definiert die Domäne der Formel des Ausdrucks. Dies ist eine Menge, welche folgende Elemente beinhaltet:
\begin{itemize}
	\item alle in der Formel vorkommende Konstanten
	\item alle Attributwerte von Relationen, welche in der Formel referenziert werden
\end{itemize}
Das Ergebnis des Ausdrucks muss nun Teilmenge dieser Domäne sein, um sicher zu stellen, dass das Ergebnis der Anfrage endlich ist.
\section{Aufgabe}
\[\{ n | \neg (n \in \text{Professoren}) \}\]
Eine unsichere Anfrage wie in Aufgabe 2, kann in der relationalen Algebra nicht konstruiert werden. Wie der Name schon sagt, beschreibt die relationale Algebra Rechenoperationen auf Relationen, welche an sich bereits endlich sind. Alle Operatoren der relationalen Algebra ermöglichen höchstens das vervielfachen der Anzahl der betrachteten Elemente, aber nie das Wachstum ins unendliche.
\pagebreak
\section{Aufgabe}
\subsection{}
\begin{lstlisting}[language=SQL]
select m.Title
from movie m, directs d, person p
where m.MovieId = d.MovieId 
	and d.PersonId = p.PersonId 
	and p.Lastname = "Murnau" 
	and p.Firstname = "Friedrich Wilhelm"
\end{lstlisting}
\subsection{}
\begin{lstlisting}[language=SQL]
select m.Title
from movie m, directs d, person p
where m.MovieId = d.MovieId 
	and d.PersonId = p.PersonId 
	and p.Lastname = 'Burton' 
	and p.Firstname = 'Tim' 
	and m.MovieId not in (
		select a.MovieId
		from acts a, person pjd
		where a.PersonId =  pjd.PersonId 
			and pjd.Lastname = 'Depp' 
			and pjd.Firstname = 'Johnny')
\end{lstlisting}
\subsection{}
\begin{lstlisting}[language=SQL]
select m.Title, c.Name, c.Location, s.Time, s.Price
from movie m, cinema c, screening s, genre g, hasGenre hg, screeningRoom sr
where s.Date = Today() 
	and s.MovieId = m.MovieId 
	and m.MovieId = hg.MovieId 
	and hg.GenreId = g.GenreId 
	and g.Name = 'Documentary' 
	and s.RoomId = sr.RoomId 
	and sr.inCinema = c.CinemaId
\end{lstlisting}
\subsection{}
\begin{lstlisting}[language=SQL]
Select a.Firstname,  a.Lastname,  d.Firstname,  d.Lastname
from person a, person d
where
	exists(
		select *
		from  directs x
		where d.PersonId = x.PersonId
		and exists(
			select *
			from acts y
			where a.PersonId = y.PersonId
				and  x.MovieId = y.MovieId
			)or  exists(
				select *
				from acts x2
				where d.PersonId = x2.PersonId
				and exists(
					select *
					from acts y
					where a.PersonId = y.PersonId
						and x2.MovieId = y.MovieId
				)	and not (
				not(d.Firstname = "Lars" and d.Lastname = "von Trier")
				and not(d.Firstname = "Michael" and d.Lastname = "Haneke")
		)
	)
)
\end{lstlisting}
\subsection{}
\begin{lstlisting}[language=SQL]
Select c.Name, c.Location, s.Price, m.Title
from cinema c, screening s, screeningRoom r, movie m
where m.MovieId = s.MovieId
	and s.RoomId = r.RoomId
	and c.Cinemaid = r.inCinema
	and exists (
		Select m.MovieId = s.MovieId
		and s.RoomId = r.RoomId
		and c.Cinemaid = r.inCinema
		and s.Date = Today() 
		and s.Time =  2100
)
\end{lstlisting}
\end{document}