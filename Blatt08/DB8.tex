\documentclass[a4paper,11pt,twoside]{scrartcl}
\usepackage[T1]{fontenc}
\usepackage{subcaption}
\usepackage[utf8]{inputenc}
\usepackage{ngerman, eucal, mathrsfs, amsfonts, bbm, amsmath, amssymb, stmaryrd,graphicx, array, geometry, color, wrapfig, float, hyperref}
\geometry{left=25mm, right=15mm, bottom=25mm}
\setlength{\parindent}{0em} 
\setlength{\headheight}{0em} 
\title{Datenbanken\\ Blatt 8}
\subtitle{Gruppe 26}
\author{Markus Vieth \and Christian Stricker}
\date{\today}
\input{../head/lstlisting.tex}
\begin{document}

\newcommand{\cor}[1]{\textcolor{red}{\textit{#1}}}
\maketitle
\cleardoublepage
\pagestyle{myheadings}
\markboth{Markus Vieth, Christian Stricker}{Markus Vieth, Christian Stricker}

\newpage

\section*{Nr.1}
\subsection*{Berechnung des Canonical Covers:}
\begin{lstlisting}
Linksreduktion von:
CD → ABF
~~~~~~~~~~~~~~~~~~~~
Prüfe ob C überflüssig ist:
Attributhülle von :
F:
CD → ABF
D → CE
F → CD
A → BED
E → BD

α - C:
D

Berechnung der Attributhülle
Erg = D
D ⊆ D ⇒ Erg = CED

E ⊆ CED ⇒ Erg = CBED

Erg = CBED
CD ⊆ CBED ⇒ Erg = ACBEDF

Erg = ACBEDF
Attributhülle = ACBEDF
ABF ⊆ ACBEDF
⇒ C ist überflüssig
++++++++++++++++++++
Prüfe ob D überflüssig ist:
Attributhülle von :
F:
D → ABF
D → CE
F → CD
A → BED
E → BD

α - D:
∅

Berechnung der Attributhülle
Erg = ∅
Attributhülle = ∅
ABF ⊈ ∅
⇒ D ist nicht überflüssig
++++++++++++++++++++
Linksreduktion von:
D → CE
~~~~~~~~~~~~~~~~~~~~
Prüfe ob D überflüssig ist:
Attributhülle von :
F:
D → ABF
D → CE
F → CD
A → BED
E → BD

α - D:
∅

Berechnung der Attributhülle
Erg = ∅
Attributhülle = ∅
CE ⊈ ∅
⇒ D ist nicht überflüssig
++++++++++++++++++++
Linksreduktion von:
F → CD
~~~~~~~~~~~~~~~~~~~~
Prüfe ob F überflüssig ist:
Attributhülle von :
F:
D → ABF
D → CE
F → CD
A → BED
E → BD

α - F:
∅

Berechnung der Attributhülle
Erg = ∅
Attributhülle = ∅
CD ⊈ ∅
⇒ F ist nicht überflüssig
++++++++++++++++++++
Linksreduktion von:
A → BED
~~~~~~~~~~~~~~~~~~~~
Prüfe ob A überflüssig ist:
Attributhülle von :
F:
D → ABF
D → CE
F → CD
A → BED
E → BD

α - A:
∅

Berechnung der Attributhülle
Erg = ∅
Attributhülle = ∅
BED ⊈ ∅
⇒ A ist nicht überflüssig
++++++++++++++++++++
Linksreduktion von:
E → BD
~~~~~~~~~~~~~~~~~~~~
Prüfe ob E überflüssig ist:
Attributhülle von :
F:
D → ABF
D → CE
F → CD
A → BED
E → BD

α - E:
∅

Berechnung der Attributhülle
Erg = ∅
Attributhülle = ∅
BD ⊈ ∅
⇒ E ist nicht überflüssig
++++++++++++++++++++

FDs nach Linksreduktion:
D → ABF
D → CE
F → CD
A → BED
E → BD
--------------------
Rechtsreduktion von:
D → ABF
~~~~~~~~~~~~~~~~~~~~
Prüfe ob A überflüssig ist:
Attributhülle von :
F - (D → ABF) ∪ (D → BF):
A → BED
E → BD
D → CE
D → BF
F → CD

α:
D

Berechnung der Attributhülle
Erg = D
D ⊆ D ⇒ Erg = CED

D ⊆ CED ⇒ Erg = CBEDF

Erg = CBEDF
Attributhülle = CBEDF
A ∉ CBEDF
⇒ A ist nicht überflüssig
++++++++++++++++++++
Prüfe ob B überflüssig ist:
Attributhülle von :
F - (D → ABF) ∪ (D → AF):
A → BED
E → BD
D → CE
D → AF
F → CD

α:
D

Berechnung der Attributhülle
Erg = D
D ⊆ D ⇒ Erg = CED

D ⊆ CED ⇒ Erg = ACEDF

Erg = ACEDF
A ⊆ ACEDF ⇒ Erg = ACBEDF

Erg = ACBEDF
Attributhülle = ACBEDF
B ∈ ACBEDF
⇒ B ist überflüssig
++++++++++++++++++++
Prüfe ob F überflüssig ist:
Attributhülle von :
F - (D → AF) ∪ (D → A):
A → BED
E → BD
D → CE
D → A
F → CD

α:
D

Berechnung der Attributhülle
Erg = D
D ⊆ D ⇒ Erg = CED

D ⊆ CED ⇒ Erg = ACED

Erg = ACED
A ⊆ ACED ⇒ Erg = ACBED

Erg = ACBED
Attributhülle = ACBED
F ∉ ACBED
⇒ F ist nicht überflüssig
++++++++++++++++++++
Rechtsreduktion von:
D → CE
~~~~~~~~~~~~~~~~~~~~
Prüfe ob C überflüssig ist:
Attributhülle von :
F - (D → CE) ∪ (D → E):
A → BED
E → BD
D → E
D → AF
F → CD

α:
D

Berechnung der Attributhülle
Erg = D
D ⊆ D ⇒ Erg = ED

D ⊆ ED ⇒ Erg = AEDF

F ⊆ AEDF ⇒ Erg = ACEDF

Erg = ACEDF
A ⊆ ACEDF ⇒ Erg = ACBEDF

Erg = ACBEDF
Attributhülle = ACBEDF
C ∈ ACBEDF
⇒ C ist überflüssig
++++++++++++++++++++
Prüfe ob E überflüssig ist:
Attributhülle von :
F - (D → E) ∪ (D → ):
D → 
A → BED
E → BD
D → AF
F → CD

α:
D

Berechnung der Attributhülle
Erg = D
D ⊆ D ⇒ Erg = ADF

F ⊆ ADF ⇒ Erg = ACDF

Erg = ACDF
A ⊆ ACDF ⇒ Erg = ACBEDF

Erg = ACBEDF
Attributhülle = ACBEDF
E ∈ ACBEDF
⇒ E ist überflüssig
++++++++++++++++++++
Rechtsreduktion von:
F → CD
~~~~~~~~~~~~~~~~~~~~
Prüfe ob C überflüssig ist:
Attributhülle von :
F - (F → CD) ∪ (F → D):
D → 
A → BED
E → BD
F → D
D → AF

α:
F

Berechnung der Attributhülle
Erg = F
F ⊆ F ⇒ Erg = DF

D ⊆ DF ⇒ Erg = ADF

Erg = ADF
A ⊆ ADF ⇒ Erg = ABEDF

Erg = ABEDF
Attributhülle = ABEDF
C ∉ ABEDF
⇒ C ist nicht überflüssig
++++++++++++++++++++
Prüfe ob D überflüssig ist:
Attributhülle von :
F - (F → CD) ∪ (F → C):
D → 
D → AF
F → C
A → BED
E → BD

α:
F

Berechnung der Attributhülle
Erg = F
F ⊆ F ⇒ Erg = CF

Erg = CF
Attributhülle = CF
D ∉ CF
⇒ D ist nicht überflüssig
++++++++++++++++++++
Rechtsreduktion von:
A → BED
~~~~~~~~~~~~~~~~~~~~
Prüfe ob B überflüssig ist:
Attributhülle von :
F - (A → BED) ∪ (A → ED):
D → 
A → ED
E → BD
D → AF
F → CD

α:
A

Berechnung der Attributhülle
Erg = A
A ⊆ A ⇒ Erg = AED

E ⊆ AED ⇒ Erg = ABED

D ⊆ ABED ⇒ Erg = ABEDF

F ⊆ ABEDF ⇒ Erg = ACBEDF

Erg = ACBEDF
Attributhülle = ACBEDF
B ∈ ACBEDF
⇒ B ist überflüssig
++++++++++++++++++++
Prüfe ob E überflüssig ist:
Attributhülle von :
F - (A → ED) ∪ (A → D):
D → 
E → BD
A → D
D → AF
F → CD

α:
A

Berechnung der Attributhülle
Erg = A
A ⊆ A ⇒ Erg = AD

D ⊆ AD ⇒ Erg = ADF

F ⊆ ADF ⇒ Erg = ACDF

Erg = ACDF
Attributhülle = ACDF
E ∉ ACDF
⇒ E ist nicht überflüssig
++++++++++++++++++++
Prüfe ob D überflüssig ist:
Attributhülle von :
F - (A → ED) ∪ (A → E):
D → 
A → E
E → BD
D → AF
F → CD

α:
A

Berechnung der Attributhülle
Erg = A
A ⊆ A ⇒ Erg = AE

E ⊆ AE ⇒ Erg = ABED

D ⊆ ABED ⇒ Erg = ABEDF

F ⊆ ABEDF ⇒ Erg = ACBEDF

Erg = ACBEDF
Attributhülle = ACBEDF
D ∈ ACBEDF
⇒ D ist überflüssig
++++++++++++++++++++
Rechtsreduktion von:
E → BD
~~~~~~~~~~~~~~~~~~~~
Prüfe ob B überflüssig ist:
Attributhülle von :
F - (E → BD) ∪ (E → D):
D → 
E → D
A → E
D → AF
F → CD

α:
E

Berechnung der Attributhülle
Erg = E
E ⊆ E ⇒ Erg = ED

D ⊆ ED ⇒ Erg = AEDF

F ⊆ AEDF ⇒ Erg = ACEDF

Erg = ACEDF
Attributhülle = ACEDF
B ∉ ACEDF
⇒ B ist nicht überflüssig
++++++++++++++++++++
Prüfe ob D überflüssig ist:
Attributhülle von :
F - (E → BD) ∪ (E → B):
D → 
A → E
E → B
D → AF
F → CD

α:
E

Berechnung der Attributhülle
Erg = E
E ⊆ E ⇒ Erg = BE

Erg = BE
Attributhülle = BE
D ∉ BE
⇒ D ist nicht überflüssig
++++++++++++++++++++

FDs nach Rechtsreduktion:
D → 
A → E
E → BD
D → AF
F → CD
--------------------
Entferne D →  weil beta leer

FDs nach dem Löschen der Form  α → ∅:
A → E
E → BD
D → AF
F → CD
--------------------

FDs nach dem Vereinigen:
F → CD
E → BD
D → AF
A → E
--------------------
Canonical Cover:
F → CD
E → BD
D → AF
A → E
\end{lstlisting}
\subsection*{Attributhülle von A}
Sei $\mathcal{F}$ das vorher berechnete Canonical Cover.
$AttrHuelle(\mathcal{F}, A):$\\
\begin{lstlisting}
Berechnung der Attributhülle
Erg = A
A ⊆ A ⇒ Erg = AE

E ⊆ AE ⇒ Erg = ABED

D ⊆ ABED ⇒ Erg = ABEDF

F ⊆ ABEDF ⇒ Erg = ACBEDF

Erg = ACBEDF
Attributhülle = ACBEDF
\end{lstlisting}
$\Rightarrow AttrHuelle(\mathcal{F}, A) = ABCDEF$
\subsubsection*{Anmerkung}
Auf weitere Berechnungen der Attributhülle wird verzichtet. Aus den vorhergegangenen Anwendungen sollte hervorgehen, dass diese von uns durchgeführt werden kann.
\subsection*{Candidate Keys}
Sei $\mathcal{F}$ das vorher berechnete Canonical Cover.\\
$AttrHuelle(\mathcal{F}, A) = ABCDEF = R \Rightarrow$ A ist ein CK\\
$AttrHuelle(\mathcal{F}, B) = \emptyset \neq R \Rightarrow$ B ist kein CK\\
$AttrHuelle(\mathcal{F}, C) = \emptyset \neq R \Rightarrow$ C ist kein CK\\
$AttrHuelle(\mathcal{F}, D) = ABCDEF = R \Rightarrow$ D ist ein CK\\
$AttrHuelle(\mathcal{F}, E) = ABCDEF = R \Rightarrow$ E ist ein CK\\
$AttrHuelle(\mathcal{F}, F) = ABCDEF = R \Rightarrow$ F ist ein CK\\
$AttrHuelle(\mathcal{F}, BC) = \emptyset \neq R \Rightarrow$ BC ist kein CK\\
Da jeder Schlüssel mit A oder D oder E oder F reduzierbar wäre und B und C und BC R nicht abdecken, sind A und D und E und F die CKs.
\end{document}